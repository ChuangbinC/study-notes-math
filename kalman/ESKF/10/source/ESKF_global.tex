% !TEX root = kinematics.tex


%\clearpage
%%%%%%%%%%%%%%%%%%%%%%%%%%%%%%%%%%%%%%%%%%%%%%%%%%%%%%%%%%%%%%
\section{使用全局角度误差的 ESKF}
\label{sec:ESKFglobal}

在本节中,我们将探讨在全局参考中定义角度误差的含义,而不是我们目前使用的局部定义。 
我们回顾第 \ref{sec:es-kinematics} 节和第 \ref{sec:fusion} 节的发展,并详细说明了与新定义相关的变化。

全局定义的角度误差 $\delta\bftheta$ 意味着一个左手侧(\emph{left hand side})的组合,即,
%
\begin{equation*}
\bfq_t = \delta\bfq\ \ot \bfq = \bfq\{\delta\bftheta\} \ot \bfq~.
\end{equation*}

为了完备性,我们注意保持角度速率向量 $\bfomega$的局部定义,即在连续时间内 $\dot\bfq=\frac12\bfq\ot\bfomega$ ,并且因此在离散时间内 $\bfq \gets \bfq\ot\bfq\{\bfomega\Dt\}$ ,不考虑全局定义的角度误差。 
这是为了方便起见,因为陀螺仪提供的角度速率的测量是在机体内,也就是说,是局部的。

%=============================================================
\subsection{连续时间的系统动力学}

%-------------------------------------------------------------
\subsubsection{真实和标称状态动力学}

真实动力学和标称动力学不涉及误差,其方程不变。

%-------------------------------------------------------------
\subsubsection{误差状态动力学}

我们先写下误差状态动力学方程,并进行评论和证明。
%
\begin{subequations}\label{equ:efullglobal}
%
\begin{align}
\dot{\delta\bfp} &= \delta\bfv \label{equ:epglobal} \\
\dot{\delta\bfv} &= -\hatx{\bfR(\bfa_m-b\bfa_b)}\delta\bftheta - \bfR\delta\bfa_b + \delta\bfg - \bfR\bfa_n \label{equ:evglobal}\\
\dot{\delta\bftheta} &= -\bfR\delta\bfomega_b - \bfR\bfomega_n \label{equ:etglobal}\\
\dot{\delta\bfa_b} &= \bfa_w \label{equ:eabglobal}\\
\dot{\delta\bfomega_b} &= \bfomega_w \label{equ:ewbglobal}\\
\dot{\delta\bfg} &= 0 ~, \label{equ:egglobal}
\end{align}%
\end{subequations}%
%
这里, 再次的, 所有方程除了 $\dot{\delta\bfv}$ 和 $\dot{\delta\bftheta}$ 之外都是平凡的。 
非平凡表达式如下所示。


\paragraph{方程 \eqRef{equ:evglobal}:线速度误差。}

我们希望确定 $\dot{\delta\bfv}$,速度误差的动力学。 
我们从以下关系开始
%
%
\begin{align}
\bfR_t &= (\bfI+\hatx{\delta\bftheta})\bfR  + O(\norm{\delta\bftheta}^2) \label{equ:Rtglobal}\\
\dot\bfv &= \bfR\bfa_\cB + \bfg~, \label{equ:vdot2global}
\end{align}%
%
其中方程 \eqRef{equ:Rtglobal} 是使用全局定义的误差 $\bfR_t$ 的小信号近似,并且在方程 \eqRef{equ:vdot2global} 中,我们引入 $\bfa_\cB$ 和 $\delta\bfa_\cB$ 作为在机体坐标系中的大信号和小信号加速度,定义于方程 \eqRef{equ:nomacc} 和 \eqRef{equ:pertacc},正如我们对局部定义的情况所做的那样。

我们以两种不同的形式 (左和右展开),继续写出 $\dot\bfv_t$ 的表达式方程 \eqRef{equ:vel} ,其中 $O(\norm{\delta\bftheta}^2)$ 这项被忽略了,
%
%
\begin{align*}
\dot\bfv+\dot{\delta\bfv} =& \eqbox{\dot\bfv_t} = (\bfI + \hatx{\delta\bftheta})\bfR(\bfa_\cB+\delta\bfa_\cB) + \bfg + \delta\bfg \\
\bfR\bfa_\cB+\bfg+\dot{\delta\bfv} =&
~~~~~~= \bfR\bfa_\cB+\bfR\delta\bfa_\cB + \hatx{\delta\bftheta}\bfR\bfa_\cB + \hatx{\delta\bftheta}\bfR\delta\bfa_\cB + \bfg + \delta\bfg 
\end{align*}%
%
将 $\bfR\bfa_\cB+\bfg$ 从左和右移动到
%
\begin{equation}
\dot{\delta\bfv} = \bfR\delta\bfa_\cB + \hatx{\delta\bftheta}\bfR(\bfa_\cB + \delta\bfa_\cB) + \delta\bfg
\end{equation}%
%
去掉二阶项,重新组织一些交叉积 (用 $\hatx{\bfa}\bfb=-\hatx{\bfb}\bfa$),我们得到
%
\begin{equation}
\dot{\delta\bfv} = \bfR\delta\bfa_\cB - \hatx{\bfR\bfa_\cB}\delta\bftheta + \delta\bfg~,
\end{equation}%
%
然后,回顾方程 \eqRef{equ:nomacc} 和 \eqRef{equ:pertacc} 并重新排列,我们得到了速度误差导数的表达式,
%
\begin{equation}
\eqbox{{\dot{\delta\bfv} = -\hatx{\bfR(\bfa_m-\bfa_b)}\delta\bftheta - \bfR\delta\bfa_b + \delta\bfg - \bfR\bfa_n }}
\end{equation}%


\paragraph{方程 \eqRef{equ:etglobal}:方向误差。}

我们先写出四元数导数的真实和标称定义,
%
%
\begin{align}
\dot{\bfq_t} &= \frac{1}{2}\bfq_t\ot\bfomega_t \\
\dot{\bfq} &= \frac{1}{2}\bfq\ot\bfomega~,
\end{align}%
%
并提醒我们使用的是全局定义的角度误差,即,
%
\begin{equation}
\bfq_t = \delta\bfq\ot\bfq~.
\end{equation}
%
正如我们对局部定义的误差情况所做的那样,我们还对大信号和小信号角度速率方程 \eqsRef{equ:nomangrate}{equ:pertangrate}进行分组。 
我们用两种不同的方法(左和右展开)继续计算 $\dot{\bfq_t}$ ,
%
%
\begin{align*}
\dot{{({\delta\bfq}\ot\bfq)}} =& \eqbox{\dot{\bfq_t}} = \frac{1}{2}\bfq_t\ot\bfomega_t \\
\dot{\delta\bfq}\ot{\bfq} + \delta\bfq\ot\dot{{\bfq}} =&~~~~~~= \frac{1}{2}\delta\bfq\ot\bfq\ot\bfomega_t \\
\dot{\delta\bfq}\ot{\bfq} + \frac{1}{2}\delta\bfq\ot\bfq\ot\bfomega =&&  
\end{align*}%
%
得到 $\bfomega_t = \bfomega + \delta\bfomega$,这就减少到
%
\begin{equation}
\dot{\delta\bfq}\ot{\bfq} = \frac{1}{2}\delta\bfq\ot\bfq\ot\delta\bfomega~.
\end{equation}
%
将左右项右乘以 $\bfq^*$,并回顾 $\bfq\ot\delta\bfomega\ot\bfq^* \equiv \bfR\delta\bfomega$,我们可以进一步展开如下,
%
%
\begin{align}
\dot{\delta\bfq} &= \frac{1}{2}\delta\bfq\ot\bfq\ot\delta\bfomega\ot\bfq^* \nonumber \\
&= \frac{1}{2}\delta\bfq\ot(\bfR\delta\bfomega) \nonumber \\
&= \frac{1}{2}\delta\bfq\ot\delta\bfomega_G~,
\end{align}%
%
用 $\delta\bfomega_G \triangleq \bfR\delta\bfomega$ 表示全局坐标系中的小信号角度速率。那么,
%
%
\begin{align}
\begin{bmatrix}
0\\\dot{\delta\bftheta}
\end{bmatrix} = \eqbox{2\dot{{\delta\bfq}}} 
&= {\delta\bfq}\ot\delta\bfomega_G \nonumber \\
&= \Omega(\delta\bfomega_G)\,\delta\bfq \nonumber \\
&= \begin{bmatrix}
0 & -\delta\bfomega_G\tr \\
\delta\bfomega_G & -\hatx{\delta\bfomega_G} 
\end{bmatrix}\begin{bmatrix}
1 \\
\delta\bftheta/2
\end{bmatrix} + O(\norm{\delta\bftheta}^2) ~,
\end{align}%
%
结果是一个标量和一个向量等式
%
\begin{subequations}
%
\begin{align}
0 &= \delta\bfomega_G\tr\delta\bftheta + O(|\delta\bftheta|^2) \\
\dot{\delta\bftheta} &= \delta\bfomega_G - \frac{1}{2}\hatx{\delta\bfomega_G}\delta\bftheta + O(\norm{\delta\bftheta}^2).
\end{align}%
\end{subequations}
%
第一个方程导致 $\delta\bfomega_G\tr\delta\bftheta = O(\norm{\delta\bftheta}^2)$,这是由二阶无穷小形成的,不是很有用。 
第二个方程在忽略所有二阶项之后得到,
%
\begin{equation}
\dot{\delta\bftheta} = \delta\bfomega_G = \bfR\delta\bfomega ~.
\end{equation}%
%
并且最后,回顾方程 %\eqRef{equ:nomangrate} and 
方程 \eqRef{equ:pertangrate},我们得到了全局角度误差的线性化动力学,
%
\begin{equation}
\eqbox{\dot{\delta\bftheta} = - \bfR\delta\bfomega_b - \bfR\bfomega_n}\ .
\end{equation}%

%=============================================================
\subsection{离散时间系统动力学}
%-------------------------------------------------------------
\subsubsection{标称状态}
标称状态方程不涉及误差,因此与局部定义的方向误差的情况相同。 

%-------------------------------------------------------------
\subsubsection{误差状态}

利用欧拉积分,我们得到以下一组差分方程,
%
\begin{subequations}
%
\begin{align}
\delta\bfp &\gets \delta\bfp + \delta\bfv\,\Dt \\
\delta\bfv &\gets \delta\bfv + (-\hatx{\bfR(\bfa_{m}-\bfa_{b})}\delta\bftheta - \bfR\delta\bfa_{b} + \delta\bfg)\Dt + \bfv_\bfi \\
\delta\bftheta &\gets \delta\bftheta - \bfR\delta\bfomega_{b} \Dt + \bftheta_\bfi \\
\delta\bfa_{b} &\gets  \delta\bfa_{b} + \bfa_\bfi \\
\delta\bfomega_{b} &\gets \delta\bfomega_{b} + \bfomega_\bfi \\
\delta\bfg &\gets \delta\bfg .
\end{align}%
\end{subequations}

%-------------------------------------------------------------
\subsubsection{误差状态 Jacobian 矩阵和扰动矩阵}

转换矩阵是通过简单检查上述方程得到的,
%
\begin{equation}
\bfF_\bfx = %\pjac{f}{\delta\bfx}{\bfx,\bfu_m} = 
\begin{bmatrix}
\bfI & \bfI\Dt & 0                             & 0               & 0                     & 0 \\
0 & \bfI    & \eqbox{-\hatx{\bfR(\bfa_m-\bfa_b)}\Dt}     & -\bfR\Dt            & 0                     & \bfI\Dt \\
0 & 0    & \eqbox{\bfI}   & 0               & \eqbox{-\bfR\Dt}                  & 0 \\
0 & 0    & 0                             & \bfI & 0                     & 0 \\
0 & 0    & 0                             & 0               & \bfI  & 0 \\
0 & 0    & 0                             & 0               & 0                     & \bfI \\
\end{bmatrix}~.
\end{equation}

我们观察到相对于局部定义的角度误差的三个变化 (比较上面 Jacobian 矩阵中的方框项与方程 \eqRef{equ:Fx_local_euler} 中的对应项);这些变化汇总在 \tabRef{tab:local_to_global}中。


在考虑各向同性噪声和 \appRef{sec:IntNoise}的扩展之后,扰动 Jacobian 矩阵和扰动矩阵保持不变。
%
\begin{equation}
\bfF_\bfi = %\pjac{f}{\bfi}{\bfx,\bfu_m} = 
\begin{bmatrix}
0 & 0 & 0 & 0 \\
\bfI & 0 & 0 & 0 \\
0 & \bfI & 0 & 0 \\
0 & 0 & \bfI & 0 \\
0 & 0 & 0 & \bfI \\
0 & 0 & 0 & 0 
\end{bmatrix}  
\quad,\quad
\bfQ_\bfi = \begin{bmatrix}
\bfV_\bfi & 0        & 0      & 0 \\ 
0      & \bfTheta_\bfi & 0      & 0 \\ 
0      & 0        & \bfA_\bfi & 0 \\ 
0      & 0        & 0      & \bfOmega_\bfi 
\end{bmatrix}~.
\end{equation}%
%


%=============================================================
\subsection{互补传感数据融合}

当考虑全局角度误差时,涉及 ESKF 机制的融合方程变化很小。 
我们通过 ESKF 校正、将误差注入标称状态和复位步骤来修正误差状态观测中的这些变化。

%-------------------------------------------------------------
\subsubsection{误差状态观测}

相对于局部误差定义的唯一区别在于观测函数的 Jacobian 块,该 Jacobian 块将方向与角度误差联系起来。 
这个新 Jacobian 块在下面被展开。

使用方程 \eqsRef{equ:quatMatProd}{equ:quatMatrix} 和一阶近似, $\delta\bfq\rightarrow \begin{bmatrix}
1 \\ \frac12\delta\bftheta
\end{bmatrix}$,四元数项 $\bfQ_{\delta\bftheta}$ 可推导如下,
%
\begin{subequations}
\begin{align}
\bfQ_{\delta\bftheta} \triangleq \pjac{(\delta\bfq\otimes\bfq)}{\delta\bftheta}{\bfq} 
&=\pjac{(\delta\bfq\otimes\bfq)}{\delta\bfq}{\bfq} \pjac{\delta\bfq}{\delta\bftheta}{\hat{\delta\bftheta}=0} \\
&= [\bfq]_R\,\frac12\begin{bmatrix}
0 & 0 & 0 \\
1 & 0 & 0 \\
0 & 1 & 0 \\
0 & 0 & 1 \\
\end{bmatrix} \\
&= \frac{1}{2}\begin{bmatrix}
-q_x &-q_y &-q_z\\
 q_w & q_z &-q_y\\
-q_z & q_w & q_x\\
 q_y &-q_x & q_w\\
\end{bmatrix}~.
\end{align}
\end{subequations}


%-------------------------------------------------------------
\subsubsection{将观测误差注入标称状态}

标称和误差状态的组合 $\bfx \gets \bfx\oplus\hat{\delta\bfx}$ 如下所示,
%
\begin{subequations}
%
\begin{align}
\bfp &\gets \bfp + \delta\bfp \\
\bfv &\gets \bfv + \delta\bfv \\
\bfq &\gets \bfq\{\hat{\delta\bftheta}\} \ot \bfq \\ 
\bfa_b &\gets \bfa_b + \delta\bfa_b \\
\bfomega_b &\gets \bfomega_b + \delta\bfomega_b \\
\bfg &\gets \bfg + \delta\bfg ~.
\end{align}%
\end{subequations}
%
其中只有用于四元数更新的公式受到影响。 
这被汇总于 \tabRef{tab:local_to_global}。

%-------------------------------------------------------------
\subsubsection{ESKF 复位}

ESKF 误差均值被重置,并且协方差被升级,根据,
%
%
\begin{align}
\hat{\delta\bfx} &\gets 0 \\
\bfP &\gets \bfG\bfP\bfG\tr
\end{align}%
%
与 Jacobian 矩阵
%
\begin{equation}
\bfG = \begin{bmatrix}
\bfI_6 & 0 & 0 \\
0 & \bfI+\hatx{\hat{\frac12\delta\bftheta}} & 0 \\
0 & 0 & \bfI_9
\end{bmatrix}
\end{equation}
%
其非平凡项扩展如下。 
我们的目标是得到新的角度误差 $\delta\bftheta^+$ 相对于旧的角度误差 $\delta\bftheta$ 的表达式。 
我们考虑这些事实:
\begin{itemize}
\item
误差复位时,真实方向不变,即 $\bfq_t^+ \equiv \bfq_t$。这就提供了:
%
\begin{equation}
\delta\bfq^+\ot\bfq^+ = \delta\bfq\ot \bfq~.
\end{equation}
%
\item
观测到的误差均值已注入标称状态 (参见方程 \eqRef{equ:quatErrorInjection} 和 \eqRef{equ:rotComposition}):
%
\begin{equation}
\bfq^+ = \hat{\delta\bfq} \ot \bfq ~.
\end{equation}
\end{itemize}

结合这两个特征,相对于旧的,我们得到了新的方向误差和观测误差 $\hat{\delta\bfq}$ 的表达式,
%
\begin{equation}
\delta\bfq^+ = \delta\bfq \ot \hat{\delta\bfq}^*  = [\hat{\delta\bfq}^*]_R\cdot\delta\bfq~.
\end{equation}
%
考虑到这点,
%
$\hat{\delta\bfq}^* \approx \begin{bmatrix}
1 \\ -\frac12\hat{\delta\bftheta}
\end{bmatrix}$,
%
上面的特征可以扩展为
%
\begin{equation}
\begin{bmatrix}
1 \\ \frac12\delta\bftheta^+
\end{bmatrix}
=
\begin{bmatrix}
1                  & \frac12\hat{\delta\bftheta}\tr \\
-\frac12\hat{\delta\bftheta} & \bfI + \hatx{\frac12\hat{\delta\bftheta}}
\end{bmatrix}
\cdot
\begin{bmatrix}
1 \\ \frac12\delta\bftheta
\end{bmatrix} + \cO(\norm{\delta\bftheta}^2)
\end{equation}
%
它给出了一个标量方程和一个向量方程,
%
\begin{subequations}
%
\begin{align}
\frac14\hat{\delta\bftheta}\tr\delta\bftheta &= \cO(\norm{\delta\bftheta}^2)\\
\delta\bftheta^+ &= -\hat{\delta\bftheta} + \left(\bfI + \hatx{\frac12\hat{\delta\bftheta}}\right)\delta\bftheta + \cO(\norm{\delta\bftheta}^2)
\end{align}%
\end{subequations}
%
其中第一个并不是很有用,因为它只是无穷小的关系。 
从第二个方程可以看出 $\hat{\delta\bftheta}^+ = 0$,这是我们对复位操作的期望。 
Jacobian 矩阵通过简单的检验得到,
%
\begin{equation}
\eqbox{\dpar{\delta\bftheta^+}{\delta\bftheta} = \bfI + \hatx{\frac12\hat{\delta\bftheta}}}~.
\end{equation}

相对于局部误差情况的差异被汇总于 \tabRef{tab:local_to_global}。


\begin{table*}[tb]
\renewcommand{\arraystretch}{1.7}
\caption{与方向误差定义相关的算法修正。}
\label{tab:local_to_global}
\vspace{1ex}
\centering
\begin{tabular}{|c|c|c|c|}
\hline
上下文 & 项 & 局部角度误差 & 全局角度误差 \\
\hline
\hline
误差组合 & $\bfq_t$ & $\bfq_t = \bfq\ot\delta\bfq$ & $\bfq_t = \delta\bfq\ot\bfq$ \\
\hline
\hline
\multirow{3}{*}{Euler 积分} & $\dparil{\delta\bfv^+}{\delta\bftheta}$ & $-\bfR\hatx{\bfa_m-\bfa_b}\Dt$ & $-\hatx{\bfR(\bfa_m-\bfa_b)}\Dt$ \\
&$\dparil{\delta\bftheta^+}{\delta\bftheta}$ & $\bfR\tr\{(\bfomega_m-\bfomega_b)\Dt\}$ & $\bfI$ \\
&$\dparil{\delta\bftheta^+}{\delta\bfomega_b}$ & $-\bfI\Dt$ &  $-\bfR\Dt$ \\
\hline
\hline
误差观测 & $
\bfQ_{\delta\bftheta}
%\dparil{(\delta\bfq\otimes\bfq)}{\delta\bftheta}
%\dparil{[\bfq]_L}{\delta\bftheta}
$ & $\frac12\begin{bmatrix}
-q_x &-q_y &-q_z\\
 q_w &-q_z & q_y\\
 q_z & q_w &-q_x\\
-q_y & q_x & q_w\\
\end{bmatrix}$ & $\frac12\begin{bmatrix}
-q_x &-q_y &-q_z\\
 q_w & q_z &-q_y\\
-q_z & q_w & q_x\\
 q_y &-q_x & q_w\\
\end{bmatrix}$ \\
\hline
误差注入 &  %$\bfq\gets\bfq\oplus\hat{\delta\bftheta}$  
& $\bfq \gets \bfq \ot \bfq\{\hat{\delta\bftheta}\}$ & $\bfq \gets \bfq\{\hat{\delta\bftheta}\} \ot \bfq$ \\ 
\hline
误差重置 & $\dparil{\delta\bftheta^+}{\delta\bftheta}$ & $\bfI - \hatx{\frac12\hat{\delta\bftheta}}$ & $\bfI + \hatx{\frac12\hat{\delta\bftheta}}$ \\
\hline
\end{tabular}
\end{table*}
